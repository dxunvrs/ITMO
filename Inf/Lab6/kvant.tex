%\documentclass[10pt, a4paper]{article}

% 1995 год, номер 4, Вавилов В., Об одной формуле Гюйгенса, страница 8

\newgeometry{
        top=0.5cm,
        bottom=0cm,
        left=1.7cm,
        right=1.8cm
    }

{\fontsize{10}{12}\selectfont

    8 \hfill К В А Н Т \quad 1995 / №4 \hfill {}
    \begin{multicols}{3}
                или, наконец, 
        \[
            l_{AC}+l_{A_1C}<\frac{2}{3}AA_1+\frac{1}{3}LL_1
        \]
        
        Умножив полученное неравенство на $n$, мы и получим неравенство (1).
        
        \vspace{0.3cm}
        \hrule
        \begin{flushleft}
            \textbf{\large{Дальнейшее исследование основного неравенства}}
        \end{flushleft}
        \hrule
        \vspace{0.3cm}
        Нами установлено, что число $\pi$ лежит в первой трети интервала 
        $(p_n, q_n)$ при всех $n \ge 3$. Для того чтобы уточнить расположение числа $\pi$ в этом новом интервале, 
        рассмотрим отношение длин интервалов $(p_n, \pi)$ и $(p,q_n)$. 
        Вычисления показывают (см. таблицы 1, 2), что это отношение длин, т.е. значения дробей
        \[
            (q_n-\pi)/(\pi-p_n), \hspace{3mm} n=3,6,12,24,
        \]
        достаточно близки к 2. На основании этих вычислений мы с большей степенью уверенности можем предположить, 
        что в действительности имеет место соотношение
        \[
            \lim_{n \to \infty} \frac{q_n-\pi}{\pi-p_n} = 2 \eqno (5)
        \]
        Для доказательства соотношения (5) заметим, что (рис.9)
        \[
            p_n=n\cdot \sin(\frac{\pi}{n}), q_n=n\cdot \tg(\frac{\pi}{n}), \hspace{1mm} n\ge3
        \]
        и, следовательно, \\
        \[
            \frac{q_n-\pi}{\pi-p_n} = \frac{1}{\cos(\frac{\pi}{n})}\cdot\frac{n\sin(\frac{\pi}{n})-\pi\cos(\frac{\pi}{n})}{\pi-n\sin(\frac{\pi}{n})}
        \]
        Для анализа полученного выражения нам понадобятся неравенства
        \[
            x-\frac{x^3}{6}<\sin(x)<x-\frac{x^3}{6}+\frac{x^5}{120}, \hspace{1mm} x>0, \eqno (6)
        \]
        значительно улучшающие известное не-
        
        \begin{center}
        \includegraphics[scale=0.7]{images/image.png}
        \end{center}
        \textit{Рис.9}
        \vspace{0.3cm} \\
        равенство $\sin(x)<x$ при $x>0$. 
        
        Чтобы доказать левое неравенство в (6), положим
        \[
            f(x)=\sin(x)-x+\frac{x^3}{6}
        \]
        Тогда имеем
        \[
            g_1(x)=f'(x)=\cos(x)-1+\frac{x^2}{2},
        \]
        \[
            g_2(x)=g'_2(x)=-\sin(x)+x
        \]
        Так как $sinx<x$ при $x>0$, получим $g_2(x)>0$ при $x>0$. Тем самым функция $g_1(x)$ возрастает при $x>0$.
        Но $g_1(0)=0$ и, следовательно, $g_1(x)>0$ при $x>0$.
        Функция $g_1(x)$ является производной для функции $f(x)$, для которой также $f(0)=0$.
        Но при $x>0$ имеем $g_1(x)>0$, поэтому функция $f(x)$ также возрастает и, следовательно,
        принимает только положительные значения, т.е. $f(x)>0$ при $x>0$, что и утверждалось. 

        Аналогично устанавливается правая часть неравенства (6), а также неравенства
        \[
            1-\frac{x^2}{2}<\cos(x)<1-\frac{x^2}{2}+\frac{x^4}{24}, \hspace{3mm} x>0 \eqno (7)
        \]
        (Докажите их самостоятельно!)

        Из неравенств (6) и (7) вытекают следующие приближенные формулы:
        \[
            \sin(\frac{\pi}{n}) \approx \frac{\pi}{n}-\frac{\pi^3}{6n^3}, \hspace{1mm} \cos(\frac{\pi}{n}) \approx 1-\frac{\pi^2}{2n^2}
        \]
        Следовательно,
        \[
            \frac{q_n-\pi}{\pi-p_n}=2\cdot(1-\frac{\pi^2}{2n^2})^{-1}.
        \]
        Что и завершает доказательство соотношения (5), так как $\frac{\pi^2}{2n^2}$ стремится к нулю при $n \to \infty$.

        Равенство (5) позволяет сделать следующий качественный вывод: число $\pi$, находясь при любом $n\ge3$ в интервале
        $(p_n,\frac{2}{3}p_n+\frac{1}{3}q_n)$, при всех достаточно больших значениях $n$ ближе к правому концу этого интервала,
        чем к левому.

        \vspace{-0.3cm}
        \begin{flushleft}
            \hrule
            \vspace{0.1cm}
            \textbf{\large{Формула Гюйгенса и ее эффективность}}
            \vspace{0.1cm}
            \hrule
        \end{flushleft}
        \vspace{-0.3cm}
        Архимед использовал для вычисления числа $\pi$ приближенную формулу $\pi \approx p_n, \hspace{1mm} n\ge3$.

        Гюйгенс в своей работе, в частности, получил другую приближенную формулу 
        $\pi=\frac{2}{3}p_n+\frac{1}{3}q_n, \hspace{1mm} n\ge3$, т.е. взял в качестве приближения для числа $\pi$
        правую часть неравенства (1).

        Большую эффективность формулы Гюйгенса по сравнению с формулой Архимеда можно обнаружить непосредственными
        вычислениями на микрокалькуляторе (см. табл. 1, 3). Отметим, что провести такие вычисления --- увлекательная и непростая задача.

        Можно сравнить эффективность формул Архимеда и Гюйгенса другим методом, не производя конкретных вычислений
        для $p_n$ и $q_n$. Можно использовать так называемые априорные оценки для точности этих формул.
    \end{multicols}

    \begin{multicols}{3}
    
        \begin{flushright}
            \textit{Таблица 1}
        \end{flushright}

        {
        \centering
        \begin{tabular}{|p{1cm}|c|c|} 
            \hline
            \hfill $n$ \hfill {} & $p_n$ & $q_n$ \\
            \hline
            \hfill 3 & 2,59807621 & 5,19615242 \\
            \hfill 6 & 3,00000000 & 3,46410161 \\
            \hfill 12 & 3,10582854 & 3,21539030 \\
            \hfill 24 & 3,13262861 & 3,15965994 \\
            \hfill 48 & 3,13935020 & 3,14608621 \\
            \hfill 96 & 3,14101825 & 3,14271459 \\
            \hfill 192 & 3,14145247 & 3,14187304 \\
            \hfill 384 & 3,14136760 & 3,14166273 \\
            \hfill 768 & 3,14158389 & 3,14161017 \\
            \hfill 1536 & 3,14159046 & 3,14159703 \\
            \hfill 3072 & 3,14159210 & 3,14159374 \\
            \hline
        \end{tabular}
        }
        
        \begin{flushright}
            \textit{Таблица 2}
        \end{flushright}

        {
        \centering
        \begin{tabular}{|p{2cm}|c|} 
            \hline
            \hfill $n$ \hfill {} & $(q_n-\pi)/(\pi-p_n)$ \\
            \hline
            \hfill 3 \hfill {} & 3,78012440 \\
            \hfill 6 \hfill {} & 2.27773383 \\
            \hfill 12 \hfill {} & 2,06336353 \\
            \hfill 24 \hfill {} & 2,01552959 \\
            \hfill 48 \hfill {} & 2,00386204 \\
            \hfill 96 \hfill {} & 2,00096924\\
            \hfill 192 \hfill {} & 2,00024098 \\
            \hfill 384 \hfill {} & 2,0006024 \\
            \hfill 768 \hfill {} & 2,00000150 \\
            \hfill 1536 \hfill {} & 2,00000746 \\
            \hfill 3072 \hfill {} & 2,00000094 \\
            \hline
        \end{tabular}
       }
        
        \begin{flushright}
            \textit{Таблица 3}
        \end{flushright}

        {
        \centering
        \begin{tabular}{|p{2cm}|c|} 
            \hline
            \hfill $n$ \hfill {} & $\frac{2}{3}p_n+\frac{1}{3}q_n$ \\
            \hline
            \hfill 3 \hfill {} & 3,464101615137 \\
            \hfill 6 \hfill {} & 3,14770538379 \\
            \hfill 12 \hfill {} & 3,142430130544 \\
            \hfill 24 \hfill {} & 3,141630505219 \\
            \hfill 48 \hfill {} & 3,141595540408 \\
            \hfill 96 \hfill {} & 3,14159283380 \\
            \hfill 192 \hfill {} & 3,141592653490 \\
            \hfill 384 \hfill {} & 3,141592654293 \\
            \hfill 768 \hfill {} & 3,141592653633 \\
            \hfill 1536 \hfill {} & 3,141592653392 \\
            \hfill 3072 \hfill {} & 3,141592653398 \\
            \hline
        \end{tabular}
        }
        
    \end{multicols}
}
